% candrakantah.tex - Example document using candrakantah.cls
\documentclass{candrakantah}
\usepackage{candrakantah}

\title{Demo Document for Candrakantah Class}
\author{John Doe}
\date{\today}

\begin{document}

\maketitle

\section{Introduction}
This document demonstrates the features of the \texttt{candrakantah} class.

\section{Defined Commands Demonstration}

\section{Lists}
\begin{itemize}
  \item First item
  \item Second item
  \item Third item
\end{itemize}

\section{Tables}
\begin{table}[h]
  \centering
  \begin{tabular}{lcr}
    \toprule
    Column 1 & Column 2 & Column 3 \\
    \midrule
    A        & B        & C        \\
    D        & E        & F        \\
    \bottomrule
  \end{tabular}
  \caption{Example Table}
\end{table}

\section{Colors}
This text is \textcolor{darkpowderblue}{dark powder blue}, and this text is \textcolor{cadmiumred}{cadmium red}.

\section{Hyperlinks}
Visit \href{https://www.example.com}{this link}.

\section{Verbatim Code}
\begin{Verbatim}
  function example() {
      console.log("Hello, world!");
    }
\end{Verbatim}

\section{Headers and Footers}
Check the document headers and footers to see their formatting.

\end{document}
